\documentclass[12pt]{article}
\usepackage[letterpaper, portrait, margin=1in]{geometry}
\usepackage{authblk}
\usepackage[style=numeric,backend=biber]{biblatex}

\input{/Users/sidbaskaran/preamble.tex}

\lstset{
	basicstyle=\ttfamily\small,
	frame=single,
	breaklines=true,
	mathescape
}

\hypersetup{
    colorlinks,
    citecolor=black,
    filecolor=black,
    linkcolor=black,
    urlcolor=black
}
\urlstyle{urlcolor=blue}

\title{Personal Finance Project}
\author[1]{Sidharth Baskaran}
\author[1]{Hamzah Rasool}
\affil[1]{Liberal Arts and Science Academy}
\date{March 29, 2022}

\begin{document}
\maketitle

\tableofcontents
\newpage

\section{Overview}
In the United States, it is a common trend for those with lower incomes to remain in that standard of living
due to an inability to save--the emphasis is placed on survival and short-term growth over long-term well-being of one and their family's future.
With this new proposal, the objective is to draft a realistic and attainable plan
to amass significant residual wealth while relying on a modest income and a frugal expenditure lifestyle.
The general approach to our financial plan involves framing our post-tax annual income expenditure primarily around
low-risk investment techniques: namely a Roth IRA account and a Series I Savings Bond, to which annual contributions are maxed-out.

Furthermore, we aim to purchase a very conservative number shares annually in high-risk sectors such as stocks
of well-known technology corporations and promising startup industries. Taking this approach, we
primarily rely on reliable investments while taking the benefits of high-risk investments, but with minimal risk.
In regards to personal expenditures necessary for a modest lifestyle, we propose to limit expenditures on luxury items
and aim to purchase used items wherever necessary, for they minimize loss when taking into consideration
capital investments with large depreciation (i.e. motor vehicles). 

Due to the high financial barrier of
entry to real estate, we aim to simply purchase a modest home when our savings afford a cash purchase.
Given that we utilize two low-risk investment vehicles, the excess wealth exceeding \$2.5 million will
safely allow this expenditure among others.

For the purpose of this study, we base simulations and cost-of-living estimates in Albequerque, New Mexico.
We utilize the tax rate of 30\% for all taxable accounts and do not account for sales tax on purchases for simplicity.
We will also utilize the 1\% inflation rate for the purchase of personal essentials.

\section{Contingencies}

In order to plan for contingency events, we propose to maintain the required \$55,000 emergency
fund throuhout the simulated lifetime of 25 to 68 years. By age 30, this fund will be established
by equal installments over each of the 5 years. After this, a smaller annual amount will be incrementally contributed to augment its growth.
In order to harness the power of compounding and minimize risk, we will utilize a high-yield investment account
paired with a savings account (The Balance). In the first five years, we contribute aggressively to each account in equal portions of \$5500.
Thus, disregarding interest we will end up with a base salary's worth of \$55000 in the emergency fund in 5 years.
The added returns will count as padding for volatility in markets.

\begin{enumerate}
    \item A savings account will be opened, and \$5500 will be contributed annually until 2027. Thereafter, \$2500 will be contributed.
    We will utilize the Marcus savings account by Goldman Sachs (NerdWallet Best Accounts), which provides an annual percentage yield (Investopedia) of 0.5\%.
    \item Likewise, \$5500 will be invested into the S\&P 500 index fund annually until 2027, with \$2500 thereafter.
\end{enumerate}

Note that we will not utilize an untaxed Roth IRA for the emergency fund in order to avoid the 10\% penalty for an unplanned withdrawal (Investopedia Roth IRA Penalty).

\section{Personal Expenditures}

budget
apartment rent
all the details

\section{Assets}

house
car

\section{Low-Risk Investments}

bonds
roth ira

\section{High-Risk Investments}

stocks

\section{Python Simulation}

python code

\end{document}